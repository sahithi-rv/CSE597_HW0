\documentclass{article}
% Change "article" to "report" to get rid of page number on title page
\usepackage{amsmath,amsfonts,amsthm,amssymb}
\usepackage{setspace}
\usepackage{Tabbing}
\usepackage{fancyhdr}
\usepackage{lastpage}
\usepackage{extramarks}
\usepackage{url}
\usepackage{chngpage}
\usepackage{longtable}
%\usepackage{subfigure}
\usepackage{soul,color}
\usepackage{graphicx,float,wrapfig}
%\usepackage{caption,subcaption}
\usepackage{enumitem}
\usepackage{morefloats}
\usepackage{multirow}
\usepackage{multicol}
\usepackage{indentfirst}
\usepackage{lscape}
\usepackage{pdflscape}
\usepackage{natbib}
%\providecommand{\e}[1]{\ensuremath{\times 10^{#1} \times}}

% In case you need to adjust margins:
%\topmargin=-0.45in      % Switch to the other top for overleaf
\topmargin=0.25in      %
\evensidemargin=0in     %
\oddsidemargin=0in      %
\textwidth=6.5in        %
%\textheight=9.75in       % play with this for overleaf
\textheight=9.25in       %
\headsep=0.25in         %

% Homework Specific Information
\newcommand{\hmwkTitle}{Introductions}
\newcommand{\hmwkDueDate}{Monday,\ August\  27,\ 2018}
\newcommand{\hmwkClass}{Homework 0}
\newcommand{\hmwkClassTime}{CSE 597}
\newcommand{\hmwkClassInstructor}{ Adam Lavely, Charles Pavloski} \newcommand{\hmwkAuthorNameb}{Sahithi\ Rampalli}
\newcommand{\hmwkNames}{svr46}

% Setup the header and footer
\pagestyle{fancy}
\lhead{\hmwkNames}
\rhead{\hmwkClass: \hmwkTitle} 
\cfoot{Page\ \thepage\ of\ \pageref{LastPage}}
\renewcommand\headrulewidth{0.4pt}
\renewcommand\footrulewidth{0.4pt}




%%%%%%%%%%%%%%%%%%%%%%%%%%%%%%%%%%%%%%%%%%%%%%%%%%%%%%%%%%%%%
% Make title
\title{\vspace{2in}\textmd{\textbf{\hmwkClass:\ \hmwkTitle}}\\\normalsize\vspace{0.1in}\small{\hmwkDueDate}\\\vspace{0.1in}\large{\textit{\hmwkClassInstructor\ \hmwkClassTime}}\vspace{3in}}
\date{}
\author{\textbf{\hmwkAuthorNameb} } % \\ \textbf{\hmwkAuthorNamea}}
%%%%%%%%%%%%%%%%%%%%%%%%%%%%%%%%%%%%%%%%%%%%%%%%%%%%%%%%%%%%%

\begin{document}
\begin{spacing}{1.1}
\maketitle

\newpage
\section{Syllabus Acknowledgement}

By turning in this assignment, I, Sahithi Rampalli, acknowledge that I have received and understand the course syllabus information available on \url{sites.psu.edu/psucse597fall2018}. 

\section{Introduction}

My name is Sahithi Rampalli.  I am a first year master student in the Computer Science and Engineering department. My programming experience includes C, C++, Java and Python languages and a few parallelization techniques using OpenMP and MPI tools. My research is partially computational in nature. 

My area of interest is Computer Architecture. Good general references in my field are \cite{Hennessy2017Computer} and \cite{david2015Computer}.  Good computational references in my field are \cite{Barnes1998Digital} and \cite{Flynn2000Comp}. 


\subsection{Accounts}

I have gotten an account on ACI using \url{https://ics.psu.edu/?page_id=57}.  My ACI username is svr46.

I have gotten an account on XSEDE using \url{https://portal.xsede.org/my-xsede?p_p_id=58&p_p_lifecycle=0&p_p_state=maximized&p_p_mode=view&saveLastPath=0&_58_struts_action=%2Flogin%2Fcreate_account}.  My username is sahrv.

I will be making my assignments available using github. My username is sahithi-rv. 

\subsection{My Course Project}

I am currently thinking about choosing XYZ as my $Ax=b$ problem for the semester project. I believe that this will be a good project because
\begin{itemize}
  \item reason A
  \item reason B
  \item reason ...
\end{itemize}


\section{HW 0 Code and Writeup}

You can get my assignment onto ACI using the command:

\begin{verbatim}
 git clone svr46@aci-b.aci.ics.psu.edu:/storage/work/s/svr46/cse597_fall2018/hw0/
\end{verbatim}

* Note, test this with us in class or with another person who isn't in the same group(s) as you.

\subsection{Program overview}

This is a serial hello world program, written in C. There is only one code file. The repository also contains the makefile for creating the executable, a readme, licensing information and the tex file for the write-up.


\subsection{Instructions for running and verifying the code}

\textbf{Creating the executable:}
\begin{verbatim}
module load gcc/7.3.1
make
\end{verbatim}

\textbf{Running the program:}
\begin{verbatim}
./svr46_helloWorld.out
\end{verbatim}

\textbf{Expected output:}
\begin{verbatim}
svr46 says "Hello, World!"
\end{verbatim}

\subsection{Instructions for compiling the write-up}

I used ACI to compile the document.  You can do this using the command:
\begin{verbatim}
./pdfmake.sh
\end{verbatim}

\section{Acknowledgements}

I would like to acknowledge Chris Blanton and Chuck Pavloski for helping formulate the homework material, and Justin Petucci and Rahim Charania for helping to make sure the permissions were set correctly for the git information.

\bibliographystyle{acm}
\bibliography{hw0_cse597_27Aug2018_svr46}

\end{spacing}

\end{document}

%%%%%%%%%%%%%%%%%%%%%%%%%%%%%%%%%%%%%%%%%%%%%%%%%%%%%%%%%%%%%}}
